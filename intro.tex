(

Over the past few years, the field of extrasolar planet (exoplanet) research
has really taken off thanks, in large part, to the exquisite time series
photometry measured by the \kepler\ Mission \citep{Borucki:2010}.
The Mission enabled the discovery of thousands of planets and planet
candidates outside the Solar System \citep{Rowe:2015}.
The zoo of planetary systems is extremely diverse---with sizes, masses, and
orbital periods spanning orders of magnitude---and the statistics are now
sufficient to test theories of planet formation and evolution.

The \kepler\ Mission has changed the face of exoplanet research because of its
photometric precision and the sheer volume of the dataset.
In order to discover small planets that serendipitously transit their host
stars, the \kepler\ spacecraft was designed to monitor the brightness of about
150,000 stars in one $10^\circ \times 10^\circ$ patch of the sky nearly
continuously---at a half-hour cadence---for more than three years with a
relative precision of a few parts-per-million for the brightest stars.
The Mission surpassed its fiducial goals and took data for over 4~years before
two of the reaction wheels used to stabilize the pointing failed in the Spring
of 2013.

Despite the fact that most planets never transit their host star---based on
geometric effects alone---and the fact that transit surveys are most sensitive
to large planets on short orbits, the discoveries made in the \kepler\ dataset
and careful characterization of the selection effects and search completeness
have enabled detailed studies of the true underlying distribution of planets
over a wide range in parameter space \citep[examples include][and
\chap{exopop} of this dissertation]{Howard:2012, Petigura:2013,
Foreman-Mackey:2014, Dressing:2015}.
These observational studies of the population of exoplanets are arguably the
ultimate goal of the \kepler\ Mission because it opens the door to direct
comparison with theories of planet formation and evolution.
These results can be used as targets for numerical simulations of planetary
systems or population level inferences can be combined with theoretical models
to \citep{Wolfgang:2014, Rogers:2015}.

One short-coming of the \kepler\ Mission was that it only targeted one field
and in that frame, the main focus was on relatively faint F, G, and K dwarf
stars.
This target selection was chosen to enable the study of long-period planets
and the discovery of Earth-sized planets orbiting Sun-like stars.
Unfortunately many of these stars and their planetary systems are not amenable
to radial velocity follow-up because the star is too faint to achieve the
required velocity precision or the expected velocity amplitude is too small to
detect.
In the Summer of 2014, the \kepler\ instrument was re-purposed and it began
taking data in a mode called \KT\ with substantial degraded pointing accuracy
\citep{Howell:2014}.
Because of technical constraints, \KT\ targets a different field in the
ecliptic plane every three months.
This means that it can target stars in different environments and focus on
gathering the census of planets orbiting bright, nearby stars.
It has been demonstrated that the data from \KT\ can reach precisions
comparable to the original Mission and that it can be used to discover
transiting exoplanets \citep[][and \chap{ketu} of this
dissertation]{Vanderburg:2014, Vanderburg:2015, Crossfield:2015,
Foreman-Mackey:2015}.
The discoveries made using \KT---and the upcoming \tess\ Mission---improve our
knowledge of the population of exoplanets, especially those planets that orbit
the cool M-stars that were not prioritized by the \kepler\ Mission.
These discoveries also present excellent targets for radial velocity follow-up
and even spectroscopic observations of their atmospheres using the planned
\project{James Webb Space Telescope}.


{Methods for detecting planets: RV, transits, direct detection,
microlensing, \etc}

{Early discoveries: first planet, hot Jupiters, first transiting
planets, \etc}

{The beginning of \kepler: first discoveries, most interesting
discoveries (most Earth-like, multis, \etc)}

One major contribution of the \kepler\ Mission was that the large number of
planet discoveries enabled statistical studies of the entire population of
planets.
Obserationally, this often manifests itself as measuring the occurrence rate
of planets as a function of their physical parameters (orbital period, mass,
radius, \etc) and correcting for selection effects and detection efficiency.

{The population of exoplanets: RV catalogs, cuts to assume complete,
power law models, Howard, Petigura, DFM, \etc}

{The distribution of planets: rare hot Jupiters, common mini-Neptunes,
Earths?, multis, inclination, \etc}

{search-characterization-population inference}

{methods for transit search: de-trending, filtering, \etc}

{characterization: mandel \& agol, kipping, carter \& winn, GPs}

{population: histograms, detection efficiency, likelihood function,
power laws, uncertainties, \etc}

{connections to planet formation, \etc}

{my contributions: search (simultaneous fitting), characterization
(GPs), population inference (hierarchical inference), \etc}

{software, \etc}

\chapname s~\chapalt{emcee} and~\chapalt{exopop} have both been refereed and
published in the astronomical literature.
\Chap{ketu} has been submitted to \emph{The Astrophysical Journal} and updated
in response to the referee's comments.
All of these \chapname s were co-authored with collaborators but the majority
of the work and writing in each \chapname\ is mine.
Here, I describe my specific contributions to each \chapname:
\begin{enumerate}

{\item For \chap{emcee}, I generalized the algorithm proposed by
\citet{Goodman:2010} through discussions with Goodman and Hogg.
I implemented the algorithm with contributions from Lang and wrote the paper
with some additions by Hogg.}

{\item For \chap{exopop}, I developed the project idea in collaboration with
Hogg and Morton.
I then implemented the project and wrote the paper with contributions from
Hogg.}

{\item Of the published \chapname s, \chap{ketu} was the most collaborative.
I developed the idea for the algorithm building on previous work with Hogg,
Wang, and Sch\"olkopf.
Using this algorithm, I wrote the code to search for transits in the
\project{K2} Campagin 1 dataset and deployed it on the NYU HPC Butinah
cluster\footnote{\url{http://nyuad.nyu.edu/en/research/infrastructure-and-support/nyuad-hpc.html}}.
I wrote the majority of the paper with Sections contributed by Montet and
Morton.}

\end{enumerate}
