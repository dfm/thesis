Over the past few years, the field of extrasolar planet (exoplanet) research
has really taken off thanks, in large part, to the exquisite time series
photometry measured by the \kepler\ Mission \todo{cite}.
The Mission enabled the discovery of over 5000 \todo{check} planets and planet
candidates outside the Solar System.
The zoo of planetary systems is extremely diverse and the statistics are now
sufficient to test theories of planet formation and evolution \todo{cite}.

{Methods for detecting planets: RV, transits, direct detection,
microlensing, \etc}

{Early discoveries: first planet, hot Jupiters, first transiting
planets, \etc}

{The beginning of \kepler: first discoveries, most interesting
discoveries (most Earth-like, multis, \etc)}

One major contribution of the \kepler\ Mission was that the large number of
planet discoveries enabled statistical studies of the entire population of
planets.
Obserationally, this often manifests itself as measuring the occurrence rate
of planets as a function of their physical parameters (orbital period, mass,
radius, \etc) and correcting for selection effects and detection efficiency.

{The population of exoplanets: RV catalogs, cuts to assume complete,
power law models, Howard, Petigura, DFM, \etc}

{The distribution of planets: rare hot Jupiters, common mini-Neptunes,
Earths?, multis, inclination, \etc}

{search-characterization-population inference}

{methods for transit search: de-trending, filtering, \etc}

{characterization: mandel \& agol, kipping, carter \& winn, GPs}

{population: histograms, detection efficiency, likelihood function,
power laws, uncertainties, \etc}

{connections to planet formation, \etc}

{my contributions: search (simultaneous fitting), characterization
(GPs), population inference (hierarchical inference), \etc}

{software, \etc}

\chapname s~\chapalt{emcee} and~\chapalt{exopop} have both been refereed and
published in the astronomical literature.
\Chap{ketu} has been submitted to \emph{The Astrophysical Journal} and updated
in response to the referee's comments.
All of these \chapname s were co-authored with collaborators but the majority
of the work and writing in each \chapname\ is mine.
Here, I describe my specific contributions to each \chapname:
\begin{enumerate}

{\item For \chap{emcee}, I generalized the algorithm proposed by
\citet{Goodman:2010} through discussions with Goodman and Hogg.
I implemented the algorithm with contributions from Lang and wrote the paper
with some additions by Hogg.}

{\item For \chap{exopop}, I developed the project idea in collaboration with
Hogg and Morton.
I then implemented the project and wrote the paper with contributions from
Hogg.}

{\item Of the published \chapname s, \chap{ketu} was the most collaborative.
I developed the idea for the algorithm building on previous work with Hogg,
Wang, and Sch\"olkopf.
Using this algorithm, I wrote the code to search for transits in the
\project{K2} Campagin 1 dataset and deployed it on the NYU HPC Butinah
cluster\footnote{\url{http://nyuad.nyu.edu/en/research/infrastructure-and-support/nyuad-hpc.html}}.
I wrote the majority of the paper with Sections contributed by Montet and
Morton.}

\end{enumerate}
